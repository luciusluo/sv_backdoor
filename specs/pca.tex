\documentclass[leqno,12pt]{article}


\usepackage{ifxetex}

% PAGE SETTINGS %

\parskip 0in
\parindent 0.32in

\usepackage[letterpaper]{geometry}
\geometry{text={5.562306in,9.0in},
headsep=0.5in,footskip=0.5in,centering}


\ifxetex
  \usepackage[T1]{fontenc}
  \usepackage[no-math]{fontspec}

  \setmainfont[Scale=1.0]{Kepler Std Light}

  \usepackage[eucal,mtphrb,mtpfrak,noamssymbols,
  	      subscriptcorrection,zswash]{mtpro2}

  \DeclareMathSizes{12}{12}{9}{7}
\else
  \usepackage[T1]{fontenc}
  \usepackage{lmodern}

  % to match mt2pro
  \usepackage{upgreek}
  \newcommand{\mbf}{\mathbf} 

  \usepackage{mathrsfs}

\fi
  
\usepackage{amssymb}
\usepackage[scr=boondox, scrscaled=1]{mathalfa}


% PACKAGES %

\usepackage{amsmath}
\usepackage{graphicx}
\usepackage{enumitem}
\usepackage{comment}
\usepackage{url}
\usepackage{marginnote}
\usepackage{booktabs}
\usepackage{bm}
\usepackage{mdframed}
\usepackage{xfrac}


% ALGORITHMS %

\usepackage{float}
\floatstyle{ruled}
\newfloat{algorithm}{hpt}{lop}
\floatname{algorithm}{Algorithm}

% THEOREMS %

\usepackage{amsthm}
\newtheoremstyle{mytheoremstyle}
{\topsep}                    % Space above
{\topsep}                    % Space below
{\slshape}                   % Body font
{0.32in}                           % Indent amount
{\bfseries}                  % Theorem head font
{.}                          % Punctuation after theorem head
{0.08in}                       % Space after theorem head
{}  % Theorem head spec (can be left empty, meaning ‘normal’)

\theoremstyle{mytheoremstyle}

\makeatletter
\@ifclassloaded{report}{
   \newtheorem{theorem}{Theorem} }
{  \newtheorem{theorem}{Theorem}[section] }
\makeatother


\newtheorem{thm}[theorem]{Theorem}
\newtheorem{proposition}[theorem]{Proposition}
\newtheorem{lemma}[theorem]{Lemma}
\newtheorem{corollary}[theorem]{Corollary}
\newtheorem{definition}[theorem]{Definition}
\newtheorem{remark}[theorem]{Remark}
\newtheorem{example}[theorem]{Example}
\newtheorem{condition}[theorem]{Condition}
\newtheorem{assumption}[theorem]{Assumption}
\newtheorem{claim}[theorem]{Claim}

\newtheorem*{theo}{Theorem}
\newtheorem*{lemm}{Lemma}
\newtheorem*{defn}{Definition}
\newtheorem*{prop}{Proposition}
\newtheorem*{assm}{Assumption}
\newtheorem*{cond}{Condition}
\newtheorem*{coro}{Corollary}
\newtheorem*{remk}{Remark}
\newtheorem*{expl}{Example}
\newtheorem*{clai}{Claim}

\renewcommand\qedsymbol{$\blacksquare$}
\renewenvironment{proof}[1][\proofname]{{\scshape #1. }}
{\qed\vspace{0.16in}}

% SECTIONS %
\usepackage{titlesec}

% section that begins a paragraph (. separator)
\titleformat{\section}[hang]
{\centering\bfseries}
{\thesection.}{0.04in}{}[]
\titlespacing{\section}{0.32in}{0.32in}{0.08in}

\titleformat{\subsection}[runin]
{\bfseries} 
{\thesubsection.}{0.04in}{}[.]
\titlespacing{\subsection}{0.32in}{0.08in}{0.08in}

% ABSTRACT

\renewcommand{\abstractname}{\bf Abstract}

\renewenvironment{abstract}
{\small
\begin{center}
  \bfseries \abstractname\vspace{0in}\vspace{0pt}
\end{center}
\list{}{
	\setlength{\leftmargin}{0.5in}
    \setlength{\rightmargin}{\leftmargin}
	\itemindent \parindent} 
  \item\relax}
{\endlist}


% TABLES AND FIGURES

\usepackage{caption}
\captionsetup[figure]{margin=0.16in,
	labelsep=period,labelfont=bf,textfont=it}
\captionsetup[table]{margin=0.16in,
	labelsep=period,labelfont=bf,textfont=it}


% REFERENCES %

\usepackage{hyperref}
\hypersetup{colorlinks=true, allcolors=deluge}

\usepackage{harvard}
\makeatletter
\@ifpackageloaded{harvard}{}
  {\newcommand{\citeasnoun}{\cite}}
\makeatother

\newcommand{\ci}{\citeasnoun}
\newcommand{\req}[1]{(\ref{#1})}

\newcommand{\hci}[2]{ \href{#2}{\ci{#1}} }
\newcommand{\hcite}[2]{ \href{#2}{\cite{#1}} }


% FRAMES %

\global\mdfdefinestyle{clean}{%
linecolor=black,linewidth=0.5pt,%
leftmargin=0.32in,rightmargin=0.32in
}

\global\mdfdefinestyle{wide}{%
linecolor=black,linewidth=0.5pt,%
leftmargin=0in,rightmargin=0in
}









%\newif\ifhide
%\hidetrue %\hidefalse
%\newcommand{\hide}[1]{\ifhide {} \else {#1} \fi} 





% SPACING/TEXT %

\newcommand{\s}[1][1]{\hspace{#1pt}}
\newcommand{\mst}{{\scalebox{0.6}[0.6]{\(\hspace{-1pt}{*}\)}}}

\newcommand{\tq}[1]{{\textquotedblleft #1\textquotedblright}}
\newcommand{\ts}[2]{{#1\textsubscript{#2}}}


% PROBABILITY %

\newcommand{\cid}{\overset{\scrD}{\rightarrow}}
\newcommand{\eqd}{\overset{\scrD}{=}}
\newcommand{\cip}{\overset{p}{\rightarrow}}
\newcommand{\cas}{\overset{a.s.}{\rightarrow}}
\newcommand{\sip}{\overset{p}{\sim}}
\newcommand{\sas}{\overset{a.s.}{\sim}}

\newcommand{\setst}{\mathbb} % set style
\newcommand{\oprst}{\textup} % operator style

\newcommand{\inds}[1]{ \mathbb{1}_{ \s[-0.5] {#1} }}
\newcommand{\indf}[1]{ \mathbb{1}_{ \s[-0.5] \{#1\} }}

\newcommand{\ms}{\scalebox{0.66}[1.0]{\( - \)}}

\newcommand{\var}{\oprst{var}}
\newcommand{\std}{\oprst{std}}
\newcommand{\cov}{\oprst{cov}}
\newcommand{\Var}{\oprst{Var}}
\newcommand{\Cov}{\oprst{Cov}}
\newcommand{\Std}{\oprst{SD}}
\newcommand{\MSE}{\oprst{MSE}}
\newcommand{\RMSE}{\oprst{RMSE} }
\newcommand{\Exp}{\oprst{E}\s[.5]}
\newcommand{\Prb}{\oprst{P}\s[.5]}
\newcommand{\Qrb}{\ensuremath{\oprst{Q}\s[.5]}}

\newcommand{\tsum}{{\textstyle\sum\nolimits}}

% SETS / MATH $

\newcommand{\nn}{\bbN}
\newcommand{\rn}{\bbR}
\newcommand{\zn}{\bbZ}
\newcommand{\qn}{\bbQ}

\newcommand{\Img}{\mathrm{Im}}
\newcommand{\Rea}{\mathrm{Re}}
\newcommand{\im}{\mathrm{i}}


\newcommand{\bst}{ \s[1.5] | \s[1.5] }
\newcommand{\cst}{ \s[0.5] : \s[0.5] }
\newcommand{\sst}{ \s[2.0] ; \s[0.5] }


\newcommand{\mat}[1]{\mbf{#1}}
\newcommand{\ones}{\rme}
\newcommand{\tr}{\textrm{tr}}   
\newcommand{\diag}{\textrm{diag}}   

% LETTER STYLES %
\usepackage{pgffor}

\foreach \x in {A,B,...,Z,a,b,...,z} 
{
  \expandafter\xdef\csname cal\x\endcsname{\noexpand 
	\ensuremath{\noexpand\mathcal{\x}}}
  \expandafter\xdef\csname scr\x\endcsname{\noexpand 
	\ensuremath{\noexpand\mathscr{\x}}}
  \expandafter\xdef\csname bb\x\endcsname{\noexpand 
	\ensuremath{\noexpand\mathbb{\x}}}
  \expandafter\xdef\csname rm\x\endcsname{\noexpand 
	\ensuremath{\noexpand\mathrm{\x}}}
  \expandafter\xdef\csname bf\x\endcsname{\noexpand 
	\ensuremath{\noexpand\mbf{\x}}}
}


% GREEK %

\newcommand{\ep}{\epsilon}


% ARROWS %

\newcommand{\upto}{ \uparrow }
\newcommand{\downto}{ \downarrow }
\newcommand{\nto}{ \nrightarrow }



% COLABORATION %

\newcommand{\astext}[1]{\delugetext{#1 -- alex}}
\newcommand{\mvtext}[1]{\delugetext{#1 -- matt}}

% LOCAL %

\newcommand{\Rcll}{C\`adl\`ag }
\newcommand{\rcll}{c\`adl\`ag }

\newcommand{\prj}[2]{\langle #1 , #2 \rangle}

\newcommand{\ip}[2]{\langle #1 , #2 \rangle}

\newcommand{\cv}{\rmd}
\newcommand{\av}{\upmu}

\newcommand{\sv}{\updelta}
\newcommand{\fv}{\upvarsigma}






\usepackage{color}
\usepackage[table]{xcolor}

\definecolor{red}{rgb}{1,0,0}
\definecolor{gray}{rgb}{0.5,0.5,0.5}
\definecolor{darkgray}{rgb}{0.4,0.4,0.4}
\definecolor{blue}{rgb}{0,0,1}
\definecolor{green}{rgb}{0,1,0}
\def\graytext#1{{\color{gray}{{#1}}\color{gray}}}
\def\darkgraytext#1{{\color{darkgray}{{#1}}\color{darkgray}}}
\def\redtext#1{{\color{red}{{#1}}\color{red}}}
\def\bluetext#1{{\color{blue}{{#1}}\color{blue}}}
\def\greentext#1{{\color{green}{{#1}}\color{green}}}


\definecolor{deluge}{RGB}{124, 113, 173}
\definecolor{bamboo}{RGB}{220, 92, 5}
\definecolor{yellow}{RGB}{255, 172, 0}
\definecolor{orange}{RGB}{255, 144, 0}
\definecolor{oyster}{RGB}{151, 139, 125}
\definecolor{coral}{RGB}{199, 186, 167}
\definecolor{downy}{RGB}{110, 197, 184}

\def\delugetext#1{{\color{deluge}{{#1}}\color{deluge}}}
\def\bambootext#1{{\color{bamboo}{{#1}}\color{bamboo}}}
\def\yellowtext#1{{\color{yellow}{{#1}}\color{yellow}}}
\def\orangetext#1{{\color{orange}{{#1}}\color{orange}}}
\def\oystertext#1{{\color{oyster}{{#1}}\color{oyster}}}
\def\coraltext#1{{\color{coral}{{#1}}\color{coral}}}
\def\downytext#1{{\color{downy}{{#1}}\color{downy}}}




\usepackage{setspace}
\usepackage{pgfornament}

\newcommand{\secbreak}{\begin{center}
\vspace{0.16in}\pgfornament[width=2in]{88}
\end{center}}

% for bold sum and product below
\usepackage{bm}

\newcolumntype{C}[1]{>{\centering\let\newline\\
	\arraybackslash\hspace{0pt}}m{#1}}
\newcolumntype{L}[1]{>{\raggedright\let\newline\\
	\arraybackslash\hspace{0pt}}m{#1}}

\renewcommand{\baselinestretch}{1.16} 

\begin{document}

% enable bold sum and product in xetex
\ifxetex
  \let\lsum\sum
  \renewcommand{\sum}{\bm{\lsum}}

  \let\lprod\prod
  \renewcommand{\prod}{\bm{\lprod}}
\else
\fi


\newcommand{\covm}{\Sigma}
\newcommand{\tcov}{\Sigma}
\newcommand{\hcov}{\bm{\hat}{\Sigma}}

{\setstretch{1.0}
\noindent
{\bf PCA model v1.0}
\noindent
\hfill {\today} \noindent
\rule{\textwidth}{0.01in}


\section{Model estimation} The following 
standadizes various asset covariance model estimates. The main 
objective an estimate of a $p \times p$ covariance matrix $\tcov$.


\begin{mdframed}[style=clean]
The input is always a data matrix of returns to assets 
over time, i.e.,
\begin{quote}
 \textsc{Input}: A $p \times n$ matrix $\rmY$ of asset returns
and a list of options.
\end{quote}
Here, $n$ is the number of observations of the return
$y \in \bbR^p$. These can be at different frequencies
(daily, weekly, monthly, etc).
\end{mdframed}

The naive (MLE) estimate of $\tcov$ is the sample covariance 
matrix $\rmS = \rmY \rmY^\top /n$. Principal Component
Analysis (PCA) regularizes this estimate by assuming
a low dimensional approximation that captures the 
maximum variance in the data.


\newcommand{\bhat}{\bm{\hat}}

PCA may be viewed as a method of estimating a 
canonical factor model (see \verb|am_standards|)
for which the covariance matrix $\tcov$ takes the form 
\begin{align}
  \tcov 
  = \Pi \Psi \Pi^\top + \Omega
  = \upsigma^2 \upbeta \upbeta^\top + \Gamma \Lambda \Gamma^\top  
  + \Omega
\end{align}


\begin{mdframed}[style=wide]

\begin{itemize}
\item[--] $\Pi = (\upbeta, \Gamma)$ 
is a $p \times (1+q)$ matrix of factor exposures. 
\item[--] There are $(q + 1)$ factors (i.e. at least one
factor).
\item[--] $\upbeta$ is a $p$ vector of market factor exposures,
\begin{align*}
  \text{requirement : } \rmm(\upbeta) = 1 \s .
\end{align*}
\item[--] $\upsigma$ is the variance of the market factor.
\item[--] $\Lambda = \diag(\uplambda^2_1, \dots, \uplambda^2_q)$
are the variances of non-market factors.
\item[--] $\Gamma = (\upgamma^1, \dots, \upgamma^q)$ is a $p \times q$ matrix of non-market
factor exposures,
\begin{align*}
  \text{requirement : }
  \rmm(\upgamma^k) \ge 0 \text{ and }
  \rms^2(\upgamma^k) = 1\s . 
\end{align*}
\item[--] $\Omega$ is the $p\times p$ matrix of specific
risks.
\end{itemize}


\end{mdframed}


\section{PCA prototype} 

The following definitions in \verb|numpydoc|
(\url{https://numpydoc.readthedocs.io/en/latest/example.html})
style define the prototypes for the input and outut
dictionaries (here treated as classes with attributes)
used to construct a PCA model.


\begin{verbatim}
Python v3.9.7
src/asset_models/PCA.py

class return_data():
    """ Specification for data input to PCA 

    Attributes
    ----------
    source : str
        Data source path and time stamp.
    n : int
        The number of observations.
    p : int
        The number of assets.
    data : numpy.array
        n x p matrix of returns to assets.
    freq : str
        Data observation frequency in {'day', 'week', 'month', 
        'quarter', 'year'}.
    start : time
        Start of observation period.
    end  : time
        End of observation period.

class pca_options(data):
    """ Options for PCA analysis for the input data
	
    Attributes
    ----------
    number_factors : int
        Number of factors if greater than 0; estimate otherwise.
    exposure_adjustments : list
        List of adjustements for exposures to factors (e.g. the
        James-Stein or correlation matrix based methods).
    variance_adjustments : list
        List of adjustements for factors variances (e.g. shrinkage
        estimators, random matrix theory based corrections.)
    specific_adjustments : list
        List of adjustments to specific risk estimates.
	
    """


class exposure_adjustment(id)
    """ Intructions to adjust factor exposures 

    Attributes
    ----------
    factor_id : int
    type : str
        Possible types are {'JS', 'COR'}
    """


class variance_adjustment(id)
    """ Intructions to adjust factor variances

    Attributes
    ----------
    factor_id : int
    type : str
        Possible types are {'RMT', 'MP'}
    """

class factor_model(id)
    """ An estimated asset model

    Attributes
    ----------
    p : int
        The number of assets.
    n : int
        The number of observations.
    q : int
        The number of factors. 
    method : str
        The method used to construct the model in {'PCA', ...}
    code : str
        Code version used to estimate the model
    options : dict
        Options passed to the method in {'pca_options', ...}
    exposures : numpy.array
        The p x q matrix of factor exposures <am_standards>
    variances : numpy.array
        The q vector of factor variances <am_standards>
    specific : numpy.array
        Either a p vector of specific risks for each asset or a
        p x p covariance matrix of the specific returns.


def pca_model(data, options)
	""" Main routing for generating a PCA model 

    Parameters
    ----------
    data : return_data
        The returns data dictionary
    options : pca_options
        Specification for a particular models (default?)
	
    Returns
    -------
    model : dict
        Estimated model.

    Notes
    -----
	
    References
    ----------
    Path to documents <am_standards>.

    Examples
    --------

\end{verbatim}



\section{PCA recipes} 

The recipe for a standardized PCA model is as follows.

\begin{mdframed}[style=clean]
\textsc{Input:} $\rmY$ and a number $q$.
\begin{itemize}
\item[\bf Step 1.] Form the sample covariance $\rmS =
\rmY \rmY^\top /n$ 
\item[\bf Step 2.] Extract $q$ eigenvectors $h^{(1)},
\dots, h^{(q)}$ from $\rmS$ along with their
eigenvalues $\scrs^2_1 \ge \dots \ge \scrs^2_q$ (largest
$q$ eigenvalues of $\rmS$ where  $n > q$).
\item[\bf Step 3] Construct $\bhat\rmB$ as follows.
The first column and $\bhat{\rmV}_{11}$ is
\begin{align*}
  \bhat{\beta} = \frac{h^{(1)}}{m(h^{(1)})}
  \s[16] \text{and} \s[16]
   \bhat{\rmV}_{11} = \scrs^2_1 \s m^2(h^{(1)}) \s .
\end{align*} 
The $k$th column of $\bhat{\rmB}$ for $1 <k \le q$
and $\bhat{\rmV}_{kk}$
is set to 
\begin{align*}
  \bhat{\gamma}^{(k)} = \frac{h^{(k)}}{m(h^{(1)})}
  \s[16] \text{and} \s[16]
   \bhat{\rmV}_{kk} = \scrs^2_k \s m^2(h^{(1)}) \s .
\end{align*} 
*Note, the normalization uses $h^{(1)}$, not $h^{(k)}$.
\item[\bf Step 4.] 
Estimate the diagonal specific return covariance as
\begin{align*}
  \bhat{\Delta} = \diag( \rmS - \bhat{\rmB} \bhat{\rmV}
  \bhat{\rmB}^\top).
\end{align*}
\item[--] {\bf Return} $(\bhat{\rmB}, \bhat{\rmV},
\bhat{\Delta})$ \\

*Note, $\rmS - \bhat{\rmB} \bhat{\rmV}
  \bhat{\rmB}^\top$ may be used as basis 
for more general estimates of a matrix $\bhat{\Omega}$,
e.g. eigenvalue truncation, sparsification, etc.
\end{itemize}
\end{mdframed}

The empirical literature is mixed on how to select the 
number of factors $q$. Even for US equitites which has
been under active investigation for $60+$ years there is 
disagreement in the empirical literature.\footnote{Various
authors propose evidence for anywhere between one and six
factors.} However, there are statistical approaches
to selecting the estimate $\bhat{q}$.  The following
recipes come from \ci{fan2020}.


\begin{mdframed}[style=clean]
\textsc{Input:} The sample covariance matrix $\rmS$.
\begin{enumerate}
\item Let $\delta_0 > 0$ be some threshold and $q_{\min}$
and $q_{\max} \le n$ be plausible lower/upper bounds on $q$. 
For eigenvaues $\scrs_1^2 \ge \cdots \ge \scrs^2_n$ of $\rmS$,
\begin{align*}
  \bhat{q} = \text{max}_{1 \le i \le q_{\max}}
  \big\{ \s i \cst \scrs^2_i -\scrs^2_{i+1}  \ge \delta_0
 \big\}
\end{align*}
\item Let  $q_{\min}$
and $q_{\max} \le n$ be plausible lower/upper bounds on $q$. 
For eigenvaues $\scrs_1^2 \ge \cdots \ge \scrs^2_n$ of $\rmS$,
we take
\begin{align*}
  \bhat{q} = \text{argmax}_{q_{\min}\le i \le q_{\max}}
  \bigg(
  \frac{\scrs^2_i -\scrs^2_{i+1}}{\scrs^2_{i+1}-\scrs^2_{i+2}}
  \bigg) \s .
\end{align*}
\item Let  $q_{\min}$
and $q_{\max} \le n$ be plausible lower/upper bounds on $q$
and set $\nu_i = \sum_{j=i+1}^n \scrs^2_j$ for eigenvaues 
$\scrs_1^2 \ge \cdots \ge \scrs^2_n$ of $\rmS$,
we take
\begin{align*}
  \bhat{q} = \text{argmax}_{q_{\min} \le i \le q_{\max}}
  \bigg(
  \frac{\log (\nu_{i-1}/\nu_i)}{\log (\nu_i/\nu_{i+1})}
  \bigg) \s .
\end{align*}
\item Let $\rmR$ be the correlation matrix for $\rmS$ (i.e.,
$\rmR = \rmD^{-1} \rmS \rmD^{-1}$ where $\rmD = \diag(\rmS)$
and let $\rho^2_1 \ge \cdots \ge \rho^2_p$ be the eigenvalues
of $\rmR$.
\begin{align*}
  \bhat{q} = \text{max}_{1 \le i \le p}
  \big\{ \s i \cst \rho^2_i > 1 \big\}
\end{align*}
\end{enumerate}
*A more advanced version of this estimator 
is in \ci{fan2020}.
\end{mdframed}

The recipe for PCA can be significantly sped up when $p$ is 
much larger than $n$, say $p \ge 2n$. The following recipe
replaced {\bf Step 2} of the PCA procedure.

\begin{mdframed}[style=clean]
\textsc{Input:} $\rmY$ and a number $q$ (assumes $p > n$)
\begin{enumerate}
\item Compute the $n \times n$ 
dual sample covariance matrix $\rmL = \rmY^\top \rmY/p$.
\item Extract $q$ eigenvectors $u^{(1)}, \dots, u^{(q)}$ of
$\rmL$
with the largest eigenvalues $\ell^2_1 \ge \cdots \ell^2_q$ and
for $1 \le i \le q$ set 
\begin{align*}
  h^{(i)} = \frac{\rmY u^{(i)}}{\ell_i \sqrt{p}} 
  \s[16] \text{and} \s[16]
  \scrs^2_i = \ell^2 p/n \s .
\end{align*}
\item[--] {\bf Return} $(\scrs^2_i, h^{(i)})_{1 \le 
i\le q}$
as the eigenpairs of $\rmS = \rmY\rmY^\top /n$.
\end{enumerate}
\end{mdframed}

The estimate of the first (market) factor $\bhat{\beta}$
provided by PCA is heavily biased. The following procedure is a
James-Stein type correction for PCA aimed to remedy this. It is
meant as an addon to {\bf Step 3} in the PCA recipe.

\begin{mdframed}[style=clean]
\textsc{Input:} The first eigenvector $h = h^{(1)}$ of $\rmS$
and eigenvalues 
$\scrs^2_1, \dots, \scrs^2_q$
(or alternatively the eigenvalues of $\rmL$ in the
spedup version
$\ell^2_1 \ge \cdots \ge \ell^2_q$).
\begin{enumerate}
\item Compute the sample average
$m(h) = \sum_{i=1}^p h_i/p$, the sample variance 
$\rms^2 (h) = \sum_{i=1}^p (h_i-m(h))^2/p$
and 
\begin{align*} 
 \rmc = 1 - \frac{\hat{\nu}^2}{\scrs_1^2 \rms^2 (h)} 
\hspace{0.08in} \text{where} \hspace{0.08in}
\hat{\nu}^2   =
\bigg( \frac{\tr(\rmS)-
(\scrs^2_1+\cdots +\scrs^2_q)}{\min(n,p)-q} \bigg) 
 \s[2]/ p \s .
\end{align*}
If the input is the eigenvalues 
$\ell^2_1 \ge \cdots \ge \ell^2_q$
of $\rmL$, we can set 
\begin{align*} 
 \rmc = 1 - \frac{\hat{\nu}^2}{\ell_1^2 \s \rms^2 (h)} 
\hspace{0.08in} \text{where} \hspace{0.08in}
\hat{\nu}^2   =
\bigg( \frac{\tr(\rmL)-
(\ell^2_1+\cdots +\ell^2_q)}{\min(n,p)-q} \bigg) 
 \s[2]/ p \s .
\end{align*}
\item Compute the corrected vector 
\begin{align*} 
 \bhat{\beta}^{\textup{JS}} = 
 \frac{m(h) + \rmc\s[1.5] (h - m(h))}
 { m(h)}  \s .
\end{align*}
*Notation, $u - x = (u_1-x,\dots, u_p-x)$
for $u \in \bbR^p$ and $x \in \bbR$.
\end{enumerate}
\end{mdframed}

Another PCA variation uses the correlation matrix 
$\rmR$ to address
the issue of bias. In the estimated model $\hcov
= \bhat{\rmB}\bhat{\rmV} \bhat{\rmB}^\top + \bhat{\Delta}$
this addresses only the estimation of the columns 
of $\bhat{\rmB}$ and should not be used to adjust the 
estimate of $\bhat{\rmV}$ in the PCA procedure. Accordingly,
it may be used to adjust all or only of the columns
of $\bhat{\rmB}$. For example, the first column may
(and perhaps should) be
James-Stein corrected as above. This also may be nicely
combined with the correlation based $q$ estimation recipe.

\begin{mdframed}[style=clean]
\textsc{Input:} $\rmY$ and a number $q$.
\begin{enumerate}
\item Compute $\rmD = \diag(\rmS)$ with $\rmS = \rmY\rmY^\top
/n$ efficienly.
\item Extract the eigevector $h$ of $\rmS$ with the largest
eigenvalue.
\item Let $\rmR$ be the correlation matrix for $\rmS$ (i.e.,
$\rmR = \rmD^{-1} \rmS \rmD^{-1}$).
\item Extract $q$ eigenvectors $v^{(1)},
\dots, v^{(q)}$ from $\rmR$ correspondind to the 
$q \le n$ largest eigenvues (sorted in decreasing order).
\item Construct $\bhat\rmB$ as follows.
The first column and $\bhat{\rmV}_{11}$ is
\begin{align*}
  \bhat{\beta} = \frac{v^{(1)}}{m(v^{(1)})}
  \s[16] \text{and} \s[16]
   \bhat{\rmV}_{11} = \scrs^2_1 \s m^2(h) \s .
\end{align*} 
The $k$th column of $\bhat{\rmB}$ for $1 <k
\le q$
and $\bhat{\rmV}_{kk}$
is set to 
\begin{align*}
  \bhat{\gamma}^{(k)} = \frac{v^{(k)}}{m(h)}
  \s[16] \text{and} \s[16]
   \bhat{\rmV}_{kk} = \scrs^2_k \s m^2(h) \s .
\end{align*} 
*Note, the normalization uses $h$, not $h^{(k)}$ nor $v^{(k)}$.
\item[\bf Step 4.] 
Estimate the diagonal specific return covariance as
\begin{align*}
  \bhat{\Delta} = \diag( \rmS - \bhat{\rmB} \bhat{\rmV}
  \bhat{\rmB}^\top).
\end{align*}
\item[--] {\bf Return} $(\bhat{\rmB}, \bhat{\rmV},
\bhat{\Delta})$ \\
\end{enumerate}
\end{mdframed}




\bibliographystyle{agsm}
\bibliography{biblio}

\end{document}
