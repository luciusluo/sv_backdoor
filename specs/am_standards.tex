\documentclass[leqno,12pt]{article}


\usepackage{ifxetex}

% PAGE SETTINGS %

\parskip 0in
\parindent 0.32in

\usepackage[letterpaper]{geometry}
\geometry{text={5.562306in,9.0in},
headsep=0.5in,footskip=0.5in,centering}


\ifxetex
  \usepackage[T1]{fontenc}
  \usepackage[no-math]{fontspec}

  \setmainfont[Scale=1.0]{Kepler Std Light}

  \usepackage[eucal,mtphrb,mtpfrak,noamssymbols,
  	      subscriptcorrection,zswash]{mtpro2}

  \DeclareMathSizes{12}{12}{9}{7}
\else
  \usepackage[T1]{fontenc}
  \usepackage{lmodern}

  % to match mt2pro
  \usepackage{upgreek}
  \newcommand{\mbf}{\mathbf} 

  \usepackage{mathrsfs}

\fi
  
\usepackage{amssymb}
\usepackage[scr=boondox, scrscaled=1]{mathalfa}


% PACKAGES %

\usepackage{amsmath}
\usepackage{graphicx}
\usepackage{enumitem}
\usepackage{comment}
\usepackage{url}
\usepackage{marginnote}
\usepackage{booktabs}
\usepackage{bm}
\usepackage{mdframed}
\usepackage{xfrac}


% ALGORITHMS %

\usepackage{float}
\floatstyle{ruled}
\newfloat{algorithm}{hpt}{lop}
\floatname{algorithm}{Algorithm}

% THEOREMS %

\usepackage{amsthm}
\newtheoremstyle{mytheoremstyle}
{\topsep}                    % Space above
{\topsep}                    % Space below
{\slshape}                   % Body font
{0.32in}                           % Indent amount
{\bfseries}                  % Theorem head font
{.}                          % Punctuation after theorem head
{0.08in}                       % Space after theorem head
{}  % Theorem head spec (can be left empty, meaning ‘normal’)

\theoremstyle{mytheoremstyle}

\makeatletter
\@ifclassloaded{report}{
   \newtheorem{theorem}{Theorem} }
{  \newtheorem{theorem}{Theorem}[section] }
\makeatother


\newtheorem{thm}[theorem]{Theorem}
\newtheorem{proposition}[theorem]{Proposition}
\newtheorem{lemma}[theorem]{Lemma}
\newtheorem{corollary}[theorem]{Corollary}
\newtheorem{definition}[theorem]{Definition}
\newtheorem{remark}[theorem]{Remark}
\newtheorem{example}[theorem]{Example}
\newtheorem{condition}[theorem]{Condition}
\newtheorem{assumption}[theorem]{Assumption}
\newtheorem{claim}[theorem]{Claim}

\newtheorem*{theo}{Theorem}
\newtheorem*{lemm}{Lemma}
\newtheorem*{defn}{Definition}
\newtheorem*{prop}{Proposition}
\newtheorem*{assm}{Assumption}
\newtheorem*{cond}{Condition}
\newtheorem*{coro}{Corollary}
\newtheorem*{remk}{Remark}
\newtheorem*{expl}{Example}
\newtheorem*{clai}{Claim}

\renewcommand\qedsymbol{$\blacksquare$}
\renewenvironment{proof}[1][\proofname]{{\scshape #1. }}
{\qed\vspace{0.16in}}

% SECTIONS %
\usepackage{titlesec}

% section that begins a paragraph (. separator)
\titleformat{\section}[hang]
{\centering\bfseries}
{\thesection.}{0.04in}{}[]
\titlespacing{\section}{0.32in}{0.32in}{0.08in}

\titleformat{\subsection}[runin]
{\bfseries} 
{\thesubsection.}{0.04in}{}[.]
\titlespacing{\subsection}{0.32in}{0.08in}{0.08in}

% ABSTRACT

\renewcommand{\abstractname}{\bf Abstract}

\renewenvironment{abstract}
{\small
\begin{center}
  \bfseries \abstractname\vspace{0in}\vspace{0pt}
\end{center}
\list{}{
	\setlength{\leftmargin}{0.5in}
    \setlength{\rightmargin}{\leftmargin}
	\itemindent \parindent} 
  \item\relax}
{\endlist}


% TABLES AND FIGURES

\usepackage{caption}
\captionsetup[figure]{margin=0.16in,
	labelsep=period,labelfont=bf,textfont=it}
\captionsetup[table]{margin=0.16in,
	labelsep=period,labelfont=bf,textfont=it}


% REFERENCES %

\usepackage{hyperref}
\hypersetup{colorlinks=true, allcolors=deluge}

\usepackage{harvard}
\makeatletter
\@ifpackageloaded{harvard}{}
  {\newcommand{\citeasnoun}{\cite}}
\makeatother

\newcommand{\ci}{\citeasnoun}
\newcommand{\req}[1]{(\ref{#1})}

\newcommand{\hci}[2]{ \href{#2}{\ci{#1}} }
\newcommand{\hcite}[2]{ \href{#2}{\cite{#1}} }


% FRAMES %

\global\mdfdefinestyle{clean}{%
linecolor=black,linewidth=0.5pt,%
leftmargin=0.32in,rightmargin=0.32in
}

\global\mdfdefinestyle{wide}{%
linecolor=black,linewidth=0.5pt,%
leftmargin=0in,rightmargin=0in
}









%\newif\ifhide
%\hidetrue %\hidefalse
%\newcommand{\hide}[1]{\ifhide {} \else {#1} \fi} 





% SPACING/TEXT %

\newcommand{\s}[1][1]{\hspace{#1pt}}
\newcommand{\mst}{{\scalebox{0.6}[0.6]{\(\hspace{-1pt}{*}\)}}}

\newcommand{\tq}[1]{{\textquotedblleft #1\textquotedblright}}
\newcommand{\ts}[2]{{#1\textsubscript{#2}}}


% PROBABILITY %

\newcommand{\cid}{\overset{\scrD}{\rightarrow}}
\newcommand{\eqd}{\overset{\scrD}{=}}
\newcommand{\cip}{\overset{p}{\rightarrow}}
\newcommand{\cas}{\overset{a.s.}{\rightarrow}}
\newcommand{\sip}{\overset{p}{\sim}}
\newcommand{\sas}{\overset{a.s.}{\sim}}

\newcommand{\setst}{\mathbb} % set style
\newcommand{\oprst}{\textup} % operator style

\newcommand{\inds}[1]{ \mathbb{1}_{ \s[-0.5] {#1} }}
\newcommand{\indf}[1]{ \mathbb{1}_{ \s[-0.5] \{#1\} }}

\newcommand{\ms}{\scalebox{0.66}[1.0]{\( - \)}}

\newcommand{\var}{\oprst{var}}
\newcommand{\std}{\oprst{std}}
\newcommand{\cov}{\oprst{cov}}
\newcommand{\Var}{\oprst{Var}}
\newcommand{\Cov}{\oprst{Cov}}
\newcommand{\Std}{\oprst{SD}}
\newcommand{\MSE}{\oprst{MSE}}
\newcommand{\RMSE}{\oprst{RMSE} }
\newcommand{\Exp}{\oprst{E}\s[.5]}
\newcommand{\Prb}{\oprst{P}\s[.5]}
\newcommand{\Qrb}{\ensuremath{\oprst{Q}\s[.5]}}

\newcommand{\tsum}{{\textstyle\sum\nolimits}}

% SETS / MATH $

\newcommand{\nn}{\bbN}
\newcommand{\rn}{\bbR}
\newcommand{\zn}{\bbZ}
\newcommand{\qn}{\bbQ}

\newcommand{\Img}{\mathrm{Im}}
\newcommand{\Rea}{\mathrm{Re}}
\newcommand{\im}{\mathrm{i}}


\newcommand{\bst}{ \s[1.5] | \s[1.5] }
\newcommand{\cst}{ \s[0.5] : \s[0.5] }
\newcommand{\sst}{ \s[2.0] ; \s[0.5] }


\newcommand{\mat}[1]{\mbf{#1}}
\newcommand{\ones}{\rme}
\newcommand{\tr}{\textrm{tr}}   
\newcommand{\diag}{\textrm{diag}}   

% LETTER STYLES %
\usepackage{pgffor}

\foreach \x in {A,B,...,Z,a,b,...,z} 
{
  \expandafter\xdef\csname cal\x\endcsname{\noexpand 
	\ensuremath{\noexpand\mathcal{\x}}}
  \expandafter\xdef\csname scr\x\endcsname{\noexpand 
	\ensuremath{\noexpand\mathscr{\x}}}
  \expandafter\xdef\csname bb\x\endcsname{\noexpand 
	\ensuremath{\noexpand\mathbb{\x}}}
  \expandafter\xdef\csname rm\x\endcsname{\noexpand 
	\ensuremath{\noexpand\mathrm{\x}}}
  \expandafter\xdef\csname bf\x\endcsname{\noexpand 
	\ensuremath{\noexpand\mbf{\x}}}
}


% GREEK %

\newcommand{\ep}{\epsilon}


% ARROWS %

\newcommand{\upto}{ \uparrow }
\newcommand{\downto}{ \downarrow }
\newcommand{\nto}{ \nrightarrow }



% COLABORATION %

\newcommand{\astext}[1]{\delugetext{#1 -- alex}}
\newcommand{\mvtext}[1]{\delugetext{#1 -- matt}}

% LOCAL %

\newcommand{\Rcll}{C\`adl\`ag }
\newcommand{\rcll}{c\`adl\`ag }

\newcommand{\prj}[2]{\langle #1 , #2 \rangle}

\newcommand{\ip}[2]{\langle #1 , #2 \rangle}

\newcommand{\cv}{\rmd}
\newcommand{\av}{\upmu}

\newcommand{\sv}{\updelta}
\newcommand{\fv}{\upvarsigma}






\usepackage{color}
\usepackage[table]{xcolor}

\definecolor{red}{rgb}{1,0,0}
\definecolor{gray}{rgb}{0.5,0.5,0.5}
\definecolor{darkgray}{rgb}{0.4,0.4,0.4}
\definecolor{blue}{rgb}{0,0,1}
\definecolor{green}{rgb}{0,1,0}
\def\graytext#1{{\color{gray}{{#1}}\color{gray}}}
\def\darkgraytext#1{{\color{darkgray}{{#1}}\color{darkgray}}}
\def\redtext#1{{\color{red}{{#1}}\color{red}}}
\def\bluetext#1{{\color{blue}{{#1}}\color{blue}}}
\def\greentext#1{{\color{green}{{#1}}\color{green}}}


\definecolor{deluge}{RGB}{124, 113, 173}
\definecolor{bamboo}{RGB}{220, 92, 5}
\definecolor{yellow}{RGB}{255, 172, 0}
\definecolor{orange}{RGB}{255, 144, 0}
\definecolor{oyster}{RGB}{151, 139, 125}
\definecolor{coral}{RGB}{199, 186, 167}
\definecolor{downy}{RGB}{110, 197, 184}

\def\delugetext#1{{\color{deluge}{{#1}}\color{deluge}}}
\def\bambootext#1{{\color{bamboo}{{#1}}\color{bamboo}}}
\def\yellowtext#1{{\color{yellow}{{#1}}\color{yellow}}}
\def\orangetext#1{{\color{orange}{{#1}}\color{orange}}}
\def\oystertext#1{{\color{oyster}{{#1}}\color{oyster}}}
\def\coraltext#1{{\color{coral}{{#1}}\color{coral}}}
\def\downytext#1{{\color{downy}{{#1}}\color{downy}}}




\usepackage{setspace}
\usepackage{pgfornament}

\newcommand{\secbreak}{\begin{center}
\vspace{0.16in}\pgfornament[width=2in]{88}
\end{center}}

% for bold sum and product below
\usepackage{bm}

\newcolumntype{C}[1]{>{\centering\let\newline\\
	\arraybackslash\hspace{0pt}}m{#1}}
\newcolumntype{L}[1]{>{\raggedright\let\newline\\
	\arraybackslash\hspace{0pt}}m{#1}}

\renewcommand{\baselinestretch}{1.16} 

\begin{document}

% enable bold sum and product in xetex
\ifxetex
  \let\lsum\sum
  \renewcommand{\sum}{\bm{\lsum}}

  \let\lprod\prod
  \renewcommand{\prod}{\bm{\lprod}}
\else
\fi

\newcommand{\res}{\upepsilon}
\newcommand{\pre}{\psi}
\newcommand{\mv}{\sigma_\rmM}
\newcommand{\mb}{\beta^{\rmM}}
\newcommand{\tcov}{\Sigma}
\newcommand{\hcov}{\bm{\hat}{\Sigma}}

{\setstretch{1.0}
\noindent
{\bf Asset modeling standards v1.1}
\noindent
\hfill {\today} \noindent
\rule{\textwidth}{0.01in}

\section{Asset return models}


\begin{mdframed}[style=clean]
Let $y = (y_1, \dots, y_p)^\top \in \bbR^p$ denote a (random) 
vector of asset returns i.e., 
each $y_i$ denotes the return
to the $i$th asset in some period.
\end{mdframed}



A general, statistical model of asset return posits that
for some predictor vector $\pre \in \bbR^q$ and 
function $f \cst \bbR^q \to \bbR^p$ there is an
approximate relationship $y \approx f(\pre)$. The error 
of this 
approximation is denoted by $\res = y - f(\pre)$, and so
\begin{align}
  y = f(\pre) + \res \s .
\end{align}
In principle, the number of dimensions, $q$, is arbitrary.
The vector $\res \in \bbR^p$ is often refered to as the 
the specific return, rather than an error residual. This
distinction refers to the decomposition of return 
into a systematic (i.e., $f(\pre)$) and specific components.

A {\bf linear asset model} posits an affine relationship
between the return and the predictor variable $\pre$.
That is, for some $p \times q$ matrix $\rmC$ we have 
$f(\pre) = \rmC \pre$. 
A {\bf nonlinear asset model} allows for determinants of 
the asset return to be driven by interactions of several 
variables (e.g., the asset returns themselves), as well as
higher order terms. These are
often implemented in the context of deep learning.
The predictor $\pre$ models the 
return of some financial risk factor in the context of a 
factor model (Sec. \ref{sec:fm}), and is the asset return
itself in the context of a graphical model (Sec. \ref{sec:gm}).
The following are references for factor, graphical and 
deep learning models of asset returns.
\begin{itemize}
 \item[--] \ci[Zangari,Equity Risk Factor Models,
Chapter 20]{litterman2004} : foundations of equity
risk models / best practices for investment management.
 \item[--] \ci{oh2021} develop a graphical model of asset
return that accomodates the analysis, estimation and management 
of equity risk.
 \item[--] Deep learning references to come (very recent).
\end{itemize}

The first two moments play a central role in the
analysis of asset return models. We denote the expected return
by $\mu = \Exp (y)$ and the covariance by
$\tcov = \Var(y)$. The latter is of particular importance
for risk management applications.

\section{Factor models} \label{sec:fm}
A factor model consists of an exposure matrix 
$\rmB \in \bbR^{q \times p}$ where $q$ is the number of 
factors and we take $\rmC = \rmB$ and $\pre = x$ to write 
the return as
\begin{align}
y = \rmB x + \res
\end{align}
The main assumption of a factor model is that
the predictor $\psi = x$ is unccorrelated with the specific
return
$\res$.  {\bf That is, we require $\Cov(x, \res) = 0$.}
The covariance matrix of 
the return $\tcov = \Var(y)$ has dimensions $p\times p$ and
 the decomposition
\begin{align} \label{covdec}
  \tcov = \rmB \rmV \rmB^\top + \Omega
\end{align}
where $\Var(x) = \rmV$ and $\Var(\res) = \Omega$ are
$q\times q$ and $p\times p$ matrices respectively. This
is a decomposition of the covariance into 
a systematic and specific risk components. 

\begin{mdframed}[style=clean]
  The $q$ columns of $\rmB$ contain factor exposures (or 
factor loadings), i.e. if $\upbeta 
= (\upbeta_1, \dots, \upbeta_p) \in \bbR^p$ denotes
a columns of $\rmB$, then $\upbeta_i$ is the exposure of the
$i$th asset to that factor (equivalently, the loading on that
factor). We assume $\rmB$ is full rank (otherwise, some
factor(s) is redundant).
\end{mdframed}

The {\bf specific risk} component 
$\Omega$ may have the following properties.

\begin{mdframed}[style=clean]
The matrix $\Omega$ is assumed to be symmetric and 
positive definite. Further, its eigenvalues are assumed to be 
bounded in $p$ (i.e., as more assets are added to the 
factor model).
A {\bf strict factor model} assumes 
$\Omega = \Delta$ is diagonal (i.e., uncorrelated
specific returns). A more general structure (e.g.,
sparsity) assumption on $\Omega$ 
implies an {\bf approximate
factor model}. 
\end{mdframed}

The {\bf systematic risk} $\rmB \rmV \rmB$ has significant
freedom in its specification/interpretation. 

\begin{mdframed}[style=clean]
The covariance of the factor returns $\rmV$ is assumed to 
be symmetric and positive definite $q\times q$ matrix. 
It is customary to assume $q$ is much smaller than $p$. Assuming 
that $\rmV$ is diagonal leads to {\bf uncorrelated
factor returns} (i.e., uncorrelated entries of $x \in \bbR^q$).
This can always
be assumed at the expense of orthogonally
transforming the matrix $\rmB$.\footnote{Let
$\rmB \rmV \rmB = \rmB \rmO \Lambda \rmO^\top \rmB
= \rmH \Lambda \rmH^\top$
for an orthogonal $\rmO$ (i.e., $\rmO^\top
\rmO = \rmO \rmO^\top = \rmI$) and $\rmH = \rmO \rmB$. 
This however does not guarantee orthogonal columns of 
$\rmH$.} The {\bf cannonical orientation} of the 
factor risk $\rmV$ and factor loadings $\rmB$ achieves
uncorrelated factor returns as well as orthogonal factors:
\begin{quote}
  Let $\Psi$ be any diagonal positive definite $q \times q$
matrix. Then, there is a nonsingular $q \times q$ matrix 
$\Phi$ such
that defining $\uppsi = \Phi^{-1} x$ and $\Pi = \rmB \Phi$ 
implies the returns satisfy
\begin{equation}  \label{canonical}
\begin{aligned}
  y &= \Pi \uppsi + \upepsilon \\
  \Var(\uppsi) &= \Psi \text{ (diagonal)}\\  
  \Var(y) &= \tcov = \Pi \Psi \Pi^\top + \Omega \\
\text{and } \Pi &\text{ has orthogonal columns.}
\end{aligned}
\end{equation}
\end{quote}
\end{mdframed}

In the {\bf cannonical orientation} the factor returns
$\uppsi = (\uppsi_1, \dots \uppsi_q)^\top$ are uncorrelated
and the factor exposures $\uppi^1, \dots, \uppi^q$
(the columns of $\Pi$) are orthogonal. That is, we have  
$\ip{\uppi^k}{\uppi^\ell} = 0$ for all  
$k \neq \ell$ and every $\ip{\uppi^k}{\uppi^k} = 
\frac{\s \upalpha_k}{\Psi_{kk}}$ for some constant
$\upalpha_k \in (0,\infty)$.
Here $\ip{\cdot}{\cdot}$ is any inner product on $\bbR^p$.
To illustrate, for a symmetric positive definite (weight)
matrix $\rmW$, we have $\ip{u}{v} = u^\top \rmW v$ 
and the standard inner product has $\rmW = \rmI$. 
The $(\upalpha_k)$
are computed via the diagonal entries of a matrix $\calA$
as follows. Let $\rmM^{\sfrac{1}{2}}$ 
denote the square-root of a matrix (i.e. 
$\rmM = 
\rmM^{\sfrac{\s 1 \s[-1]}{2}}
\rmM^{\sfrac{\s 1 \s[-1]}{2}}$) and let
\begin{align}
  \rmV^{\sfrac{\s 1\s[-1]}{2}} \rmB^\top \rmW \rmB 
\rmV^{\sfrac{\s 1\s[-1]}{2}} = \rmO \calA \rmO^\top
\end{align}
where the right side is the eigen-decomposition of the 
matrix on the left side (i.e., the $(\upalpha_k)$ are
the eigenvalues of this matrix). Further, $\Phi
= \rmV^{\sfrac{\s1\s[-1]}{2}} 
\rmO^\top \Psi^{\ms\sfrac{1\s[-1]}{2}}$ in 
$\req{canonical}$.
Then, orthonormal factor exposures  
(i.e., $\ip{\uppi^k}{\uppi^k} = 1$) may be achieved by requiring the factor variances 
to satisfy $\Psi_{kk} = \upalpha_k$. Alternatively, one
can require unit factor variances (i.e., 
$\Psi = \rmI$) at the expense of
only orthogonal factor exposures.


\begin{mdframed}[style=clean]
   It is empirical fact that most asset returns data exhibits
a factor for which the exposures are mostly all of the same
sign and the variance of this factor is large (often 
the largest). This risk factor is refered to as the 
{\bf market}. As an example in the Barra MSCI US equity
model the exposures to the market factor are set as 
the isometric vector
\begin{align}
 \rme = (1, \dots, 1)^\top \in \bbR^p
\end{align}
i.e., ever asset has unit exposure to the market. The volatility
(square-root of the variance) of this factor is typically 
assumed to be around $16\%$ annualized. In Barra models
the factor return variances are typically computed by 
regressing asset returns onto fixed factor exposures such as
$\rme$. The other Barra factors have exposures designed to 
have a zero average exposure. Barra factor exposures
are not orthogonal.
\end{mdframed}

The {\bf canonical factor model}, defined as having uncorrelated
factor returns and orthogonal factors per $\req{canonical}$, may be written to 
highlight the presence of the {\bf market factor} by decomposing
the covariance $\Sigma = \Pi \Psi \Pi^\top + \Omega$ as follows.
\begin{align}
  \Sigma = \upsigma^2 \upbeta  \upbeta^\top  
  + \Gamma \Lambda \Gamma + \Omega
\end{align}
for market exposures $\upbeta \in \bbR^p$, market variance
$\upsigma^2$, a $p \times (q-1)$ matrix $\Gamma$ of nonmarket
factor exposures with the $(q-1) \times (q-1)$
covariance matrix $\calA$ of the returns of these factors,
and the specific risk matrix $\Omega$. Here, the exposure
matrix $\Pi$ is $\Gamma$ with an additional column $\upbeta$
and covariance $\Psi$ is $\Lambda$ with an additional
diagonal element $\upsigma^2$.

\begin{mdframed}[style=clean]
  The relative sizes of the variances and 
exposures to factors is unidentifiable. For example 
$\upsigma^2 \beta \beta^\top = (\upsigma/\rmc)^2
(\rmc \s \upbeta) (\rmc\s \upbeta)^\top$  for any $\rmc \neq 0$.
Thus some canonical convention is needed to standardize
the sizes of the variances and exposures relative to one
another. Due to the interpretation of $\upbeta$ as the
market exposures we adopt the following convention:\footnote{Due
to the orthogonal property of the canonical factor model
only one factor is expected to be market-like, i.e., large
variance and exposures of mostly the same sign.}
\begin{quote}
  The average market exposure is standardized to be one
(which fixes the size of the market variance). Specifically,
for some weights $w \in \bbR^p$ with 
$\tsum_{i=1}^p w_i = 1$ and every $w_i \ge 0$,
\begin{align}
\rmm (\upbeta) = \tsum_{i=1}^p w_i \beta_i = 1 \s .
\end{align}
The standard choice is every $w_i = 1/p$ (unweighted mean).

The nonmarket factors in $\Gamma \Lambda \Gamma^\top$ 
may have average exposure equal to 
zero (the exposures $\Gamma$ have mixed signs) and for this
reason we standardize their variation instead (fixing
the $\Lambda$). Specifically,
each column $\upgamma$ of $\Gamma$ is taken to have
\begin{equation}
\begin{aligned}
 \rms^2(\upgamma) &= \tsum_{i=1}^p w_i (\upgamma_i - 
 \rmm(\upgamma))^2 = 1 , \\
 \rmm(\upgamma) &= \tsum_{i=1}^p w_i \upgamma_i \ge 0 \s .
\end{aligned}
\end{equation}
for any weight vector $w \in \bbR^p$ as above.
\end{quote}
\end{mdframed}

To get some intuition for these normalizations note that
for any vector $v \in \bbR^p$ and equal weights $w_i =1/p$ 
we have the relation the the lengh $|v| = \sqrt{\ip{v}{v}}$,
\begin{align}
 |v| &= \sqrt{p} ( \rmm^2(v) + \rms^2(v) ) \s .
\end{align}
In this way, all factor exposures are normalized to have
a lengh proportional to $\sqrt{p}$ and either the mean
$\rmm(v)$ or variation 
$\rms(v)$ are standardized to unity (for $\upbeta$ or 
$\upgamma$).


\section{Graphical models.}\label{sec:gm} 
In this model the predictor
$\pre = y$ and we write $\rmC = \rmA$ so,
\begin{align}
  y = \rmA y + \res
\end{align}
Here, $p = q$ so that $\rmA$ is a $p \times p$ matrix 
with a zero diagonal (i.e. no asset is allowed to predict
itself). The main assumption (can always be satified) is
\begin{align}
  \Cov (\rmA y , \res) = 0 \s .
\end{align}
The second main assumption is that:
\begin{itemize}
 \item[--] $\rmA$ is a sparse matrix.
\end{itemize}
This requirement may be in place for several reasons. It
can be justified from a financial perspective, from an
 estimation perspective, but also to maintain
a consistency with factor modeling. In a factor model
the number of paramers that are estimates in much smaller
than $p^2$. For example, for a strict factor model, only
$p \times q + q + p$ parameters are estimated and $q$ 
is assumed much smaller than $p$. This places a limit on the 
number of non-zero entries of $\rmA$.


The covariance decomposition in a factor model takes the form
\begin{align}
  \tcov = \rmA \tcov \rmA^\top + (\rmI + \rmA) \rmD
\end{align}
where $\rmD = \Cov(y,\res)$ which is a diagonal matrix. This
decomposition mirrors that in $\req{covdec}$ and attempts to
decompose the covariance into
\begin{itemize}
\item[--] $\rmA \tcov \rmA^\top$ -- the variance explained by
the asset returns themselves. This should be compared to
$\rmB \rmV \rmB^\top$, the variance explained by the
factors.
\item[--] The remaining variance $(\rmI + \rmA)\rmD$ 
(not explained by the asset returns).
\end{itemize}

One implication of sparsity on $\rmA$ is that the precision
matrix $\rmK = \tcov^{-1}$ is also sparse. In a graphical 
model we have the identity
\begin{align}
 \rmK = \rmD^{-1} \Omega \rmD^{-1}
\end{align}
where $\Omega = \Var(\res)$, which we see is also a sparse
matrix.





\bibliographystyle{agsm}
\bibliography{biblio}

\end{document}
